\section{Setup}

The column mode of the the Planet Simulator is an integral part 
of the full Planet Simulator and not a stand alone model. 
The advantage of this approach is that all options and modules 
available in the full model are automatically included in
the column mode and that no extra maintenance is necessary. 

Technically this is realized by switching off horizontal 
advection and diffusion which leaves us with an independent
column for each grid point. The inclusion of additional options
for boundary conditions allows for runs with synchronized columns.   

Running the Model at the lowest resolution (T1) with  $8$ 
synchronized columns is efficient enough to
run very fast column integrations. Using a standard setup
with $10$ atmospheric layers and a mixed layer ocean one 
can simulate more than $33500$ years per day on a single 
processor PC (3.3 GHZ CPU).

To make full use of the computer power one can setup an ensemble of 
columns by specifying boundary conditions for every grid point 
separately.

\subsection{Basic switches for column setup}

We introduce the macro switch {\bf COLUMN} in the namelist {\it inp}
which is part of the namelist $\mathrm{ \bf puma{\_}namelist}$.
By setting {\bf  COLUMN = 1} a default column mode is initialized by
setting {\bf YMODE = "Column"}, {\bf KICK = 0}, {\bf NADV = 0}
and {\bf NHORDIF = 0}. One can customize the column setup by keeping
the default value {\bf  COLUMN = 0} and by setting the other switches
individually. For more details see the table {\it inp}
in appendix \ref{Namelist}.  

\subsection{Boundary Conditions and forcing}

For the T1 truncation the following lower boundary conditions are specified
by external fields: The land sea mask 
($\mathrm{N002\_surf\_0172.sra}$), 
the surface geopotential ($\mathrm{N002\_surf\_0129.sra}$)
and the surface temperature ($\mathrm{N002\_surf\_0169.sra}$).
The other fields are set by default within the model. Some can be
set by namelist parameters (see description of standard model).

The surface fluxes of heat and moisture in the 
column mode can be influenced by the switch {\bf ZUMIN} 
in the namelist {\it fluxpar} which sets the surface wind 
speed entering the bulk exchange scheme. The default value 
is set to 1 m/s.

Keeping the standard settings the columns will be forced by the solar
forcing corresponding to the grid point where the column is located.
For the T1 truncation this mean that the columns are located approximately
at gaussian latitudes $\pm 35.26^\circ$. The solar forcing corresponds
to the default of the full model. A daily mean insolation is
used with an annual cycle. The solar forcing is also influenced
by the climatological ozon distribution which by default also has
an annual cycle.


\section{Graphical User Interface (GUI)}

To visualize the time evolution of the column model a vertical Hovmoeller 
plot has been added to the GUI (picture type: ISOCOL).
The sounding device can be also used to visualize the vertical profiles at 
an arbitrary grid point of the gaussian grid in the full model. 
By clicking the window the sounding goes from grid point to grid point 
in meridional direction. The longitude can be selected by the switch 
{\bf sellon}, which is a parameter of the {\bf inp} namelist in 
$\mathrm{\bf puma\_namelist}$.  
Using the template $\mathrm{GUI\_sounding.cfg}$ in folder 
{\bf plasim/dat} the GUI is configured for soundings. 
In this case {\bf sellon} can be modified in the control window. 
For more details see chapter \ref{chap_GUI}. 
